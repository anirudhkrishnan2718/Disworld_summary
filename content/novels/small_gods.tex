\section{Small Gods}


\subsection{}
\subsubsection{\Gls{Bru} hears a phantom voice in the Citadel of Kom}
In the Monastery of Time, the \Gls{AbbotH} talks to \Gls{Sweep} about the need to monitor the
small Kingdom of Omnia, where the next prophet of \Gls{Gom} is expected to show up. In a temple,
novice monk \Gls{Bru} hears a disembodied voice whisper to him, and runs away scared of demons.
He reports the strange occurrence to \Gls{Nhum}, the Master in charge of training novices, and
realizes too late that the voice did not actually speak any words.

\subsubsection{\Gls{Vorbis} participates in the torture of his former secretary for espionage}
\Gls{Gom} tries to crawl to \Gls{Sweep}, who is also a monk in the Citadel of Kom, and fails to
grab his attention. \Gls{Bru} is chided gently by \Gls{Nhum}, and sent back to his garden duty with
a warning not to heed any such voices whispering in his ears. Meanwhile, Deacon \Gls{Vorbis}, the
head of the Church's Quisition, in charge of catching and punishing heretics, tries to extract
the names of the traitors to whom his former secretary fed secrets.

\subsubsection{\Gls{Gom} asks \Gls{Bru} to bring him the highest-ranked priest in the Church}
\Gls{Vorbis} asks for the secretary to be kept alive and tortured, when he professes faith in a
different religion, in defiance of the Church's teachings. \Gls{Gom} asks \Gls{Bru} to fetch the
ranking priest in the Citadel, leaving him terrified at the prospect of taking the matter to his
immediate superior \Gls{Nhum}. \Gls{Bru} thinks back to his grandmother, whose own faith in
\Gls{Gom} had seeped down to him as a child, and recites more holy verses to calm himself down.

\subsubsection{\Gls{Vorbis} wishes to prosecute the propagation of his religion by the sword}
In an off-the-books meeting with \Gls{Vorbis}, General \Gls{Friit} who is in charge of the church's
military talks about the massive losses against the Ephebians. \Gls{Drunah}, the secretary in charge
of the council of High Priests also suggests prudence in the punishment of Ephebe for their insult
to \Gls{Murd}. \Gls{Vorbis} shows them the blasphemous document written by some scholar in Ephebe,
that talks about \Gls{Atuin} holding up the Disc, totally unmindful of its accuracy.

\subsubsection{\Gls{Vorbis} decides to head to Ephebe undercover as a diplomat}
\Gls{Vorbis} chooses to head the delegation of High Priests to be sent to Ephebe as diplomats, and
vows to purse the military conquest of every neighbouring kingdom that believes in other gods.
Meanwhile, \Gls{Bru} fetches \Gls{Nhum} from his chamber at the novice dormitories, and asks him to
come see the turtle in the garden.

\subsubsection{The followers of \Gls{Didact} plot the takeover of the Citadel}
Deep in the cellars of the Citadel, a gruop of heathen believers in \Gls{Atuin} worry about the
safety of \Gls{Didact}, the Ephebian philosopher who founded the new religion. Some of them wish to
kill \Gls{Vorbis} in secret, while others insist on biding their time until they can confront him
head-on.

\subsubsection{\Gls{Nhum} dismisses \Gls{Bru}'s claims about the talking tortoise being \Gls{Gom}}
\Gls{Nhum} looks patiently as \Gls{Bru} talks to a small yellow tortoise with one missing eye, and
wonders if the boy has lost his mind. He picks up the tortoise and tells \Gls{Bru} that it will make
for a good dinner. \Gls{Bru} is worried to hear \Gls{Gom} cursing \Gls{Nhum} in his mind, and
realizes that the words are reaching his mind telepathically.

\subsubsection{\Gls{Bru} saves \Gls{Gom} from being made into a soup}
\Gls{Bru} rescues \Gls{Gom} from the communal kitchen, believing it to be one of the spirits
mentioned in the Holy Books. \Gls{Gom} is alarmed to learn that heretics are regularly burned at
the stake by the Quisition, as \Gls{Bru} shows him the huge statue of \Gls{Gom} in the form of a
bull. \Gls{Gom} curse singes \Gls{Bru}'s eyebrow, since his power is so diminished he cannot call
forth anything bigger than a spark of lightning.

\subsubsection{\Gls{Drunah} and \Gls{Friit} try to gauge each other's interest in the \Gls{Atuin}
    Movement}
\Gls{Drunah} and \Gls{Friit} talk privately at the rooftops of the Citadel, unsure if the other is a
part of the secret society of \Gls{Atuin} believers. \Gls{Friit} makes a roundabout reference to
a coup brewing within the Citadel, and finds \Gls{Drunah} receptive to his analogy. \Gls{Bru} is
alarmed to learn that \Gls{Gom} has no memory of ever meeting any of the Great Prophets except for
\Gls{Ossory}, and no knowledge of his religion's teachings.

\subsubsection{\Gls{Gom} gives accurate information about \Gls{Bru}'s childhood}
\Gls{Gom} searches his memory, and correctly recits a punishment that \Gls{Bru} had suffered for a
crime he did not commit, and of how he had cursed his grandmother for it. \Gls{Bru} is convinced
that the tortoise is indeed \Gls{Gom}, and wonders who is listening to the prayers of the faithful
if \Gls{Gom} is incarnated as a small animal on the Disc.

\subsubsection{\Gls{Vorbis} is impressed by \Gls{Bru}'s virtues, as explained by \Gls{Nhum}}
\Gls{Bru} faints when \Gls{Vorbis} walks into his garden on his daily patrols, with \Gls{Gom}
hissing loudly nearby. \Gls{Vorbis} sets the tortoise on its back, with pebbles to stop it from
flipping itself, as a form of amusement. \Gls{Vorbis} learns from \Gls{Nhum} that \Gls{Bru} is a
loyal but unintelligent novice, who is very good at listening to every word he is told, and
retaining all of it. He asks for \Gls{Bru} to be sent to his office in the cellars, for a new
job appointment.

\subsubsection{\Gls{Sweep} rescues \Gls{Gom} from certain death}
\Gls{Gom} struggles to turn himself upright, wondering how he happened to land on a compost heap
when the eagle dropped him from a great height. \Gls{Sweep} helps him out, before leaving without a
word. Meanwhile, Sergeant \Gls{Simony} receives the message from the \Gls{Atuin} Movement, putting
him in charge of bringing \Gls{Didact} to the Citadel to complete the coup of \Gls{Vorbis}.

\subsubsection{\Gls{Gom} makes it to the Central Prayer Square in the Citadel}
\Gls{Gom} decides to seek out the High Priest on his own, and waddles into the corridors of the
Citadel, after a hearty meal in the garden. \Gls{Gom} gets a glimpse into the underground torture
chambers of the Quisition, when he looks through a grill on the way to the Central Prayer Square.
He manages to find an empty space in between the supplicants thronging the Square, and is horrified
when an eagle takes flight from the large statue at the shrine, intent on eating him.

\subsubsection{\Gls{Bru} receives a new assignment from \Gls{Vorbis} after demonstrating his memory}
\Gls{Bru} has a hood pulled over his head when he arrives at \Gls{Vorbis}'s office, and is taken to
an adjoining room. Once there, he recalls the contents of the office in great detail, to the
astonishment of \Gls{Vorbis} and his two colleagues. \Gls{Vorbis} is happy to see his colleagues
impressed, and tells \Gls{Bru} that he will accompany the diplomats going to Ephebe the next day.
\Gls{Bru} is then flummoxed when \Gls{Vorbis} asks him to forget the meeting, but plays along just
to be safe.

\subsubsection{\Gls{Gom} tries desperately to hide from the eagle in the Prayer Grounds}
\Gls{Gom} realizes that he cannot reach the High Priest on his own, and calls out for \Gls{Bru}'s
help. \Gls{Bru} looks for \Gls{Sweep} to tell him how to tend the garden in his absence, and sees
the bonsai mountain that the old man has spent crafting in his free time. \Gls{Gom} manages to
reach out to him telepathically, and demands to be picked up. The eagle intent on eating \Gls{Gom}
is frightened by trumpets suddenly screaming throughout the Citadel.

\subsubsection{\Gls{Bru} reunites with \Gls{Gom}, and then runs into \Gls{Dhblah}}
\Gls{Bru} runs into the \Gls{Cenobi}'s evening procession across the Prayer Grounds, and is escorted
away from the path by some guards. He then finds \Gls{Gom} at the bull statue at the center of the
Grounds, and picks up the tortoise just as \Gls{Dhblah} comes around with his dubious assortment
of sweet treats for sale. \Gls{Dhblah} mentions the peace mission to Ephebe, leaving \Gls{Bru}
surprised at how quickly news spread around the Citadel.

\subsubsection{\Gls{Gom} asks \Gls{Bru} to take him along to Ephebe}
\Gls{Gom} realizes that the rock-solid belief that \Gls{Bru} has in him is the only thing keeping
him from extinction. He insists on being taken along to Ephebe, just so he can keep an eye on his
one loyal worshipper, while thinking about revenge against \Gls{Vorbis} for leaving him to die.
Meanwhile, \Gls{Friit} gets drunk off his secret stash of liquor, and decides to kill \Gls{Vorbis}
to free the Church from his clutches.

\subsubsection{\Gls{Bru} assembles for the trip to Ephebe at dawn}
\Gls{Vorbis} greets \Gls{Friit} at the front door of his chamber, with two of his torturers.
The next morning, \Gls{Bru} packs \Gls{Gom} up into a small wooden box, along with some food for the
journey, and waits where \Gls{Vorbis} had asked him to. He notices some soldiers in plainclothes
readying a large group of camels, and is warned by \Gls{Gom} not to enquire any further.

\subsubsection{\Gls{Friit} crosses the desert of judgment in his afterlife}
\Gls{Vorbis} shows up to arrange a mule for \Gls{Bru}, and tells Sergeant \Gls{Simony} that
\Gls{Friit} will not be able to accompany them because of some personal emergency. \Gls{Friit}
wakes up as a spirit, and finds \Gls{Death} waiting for him in the afterlife, which is a desert to
be crossed on foot per Omnian belief.

\subsubsection{\Gls{Bru} notices the undercover soldiers take the large herd of camels into the
    desert}
\Gls{Bru} notices the herd of camels with water-packs being led by some Omnian soldiers into the
deep desert, as the diplomats continue on through the well-worn route. \Gls{Vorbis} is pleased to
see more feats of recollection by \Gls{Bru}, and asks him once again to put the undercover
soldiers out of his mind.

\subsection{}
\subsubsection{\Gls{Bru} is uncomfortable with the open sea}
\Gls{Bru} reaches the ship waiting to take them to Ephebe, and notices that everyone else is busy
boarding their horses. \Gls{Gom} complains about the lettuce leaves in his box starting to rot, as
\Gls{Bru} gets seasick. The sailors are amused by his fear of the open water, as the two-day voyage
starts.

\subsubsection{\Gls{Vorbis} asks the ship's captain to violate a deeply held superstition}
\Gls{Vorbis} forces the captain of the ship to kill a dolphin using his harpoon, in direct violation
of popular superstition, that considered them the reincarnation of dead sailors. \Gls{Gom} tells
\Gls{Bru} a lie about why he got stuck in the body of a tortoise, while hiding the truth that he is
almost extinct due to a lack of belief. With \Gls{Bru} fast asleep, \Gls{Gom} senses the coming
sea-storm as a result of the dolphin slaughter, and decides to talk to the \Gls{QSea}.

\subsubsection{\Gls{Gom} asks \Gls{Vorbis} what to do about the flashing signal from the desert}
\Gls{Gom} asks the \Gls{QSea} to save the ship, just as the sailors decide to offer \Gls{Bru} up as
human sacrifice for the sins of the captain. The storm dissipates, making the sailors cower in
terror from \Gls{Bru}, leaving the rest of the voyage uneventful. When \Gls{Bru} sees flashing
lights from the coast, \Gls{Vorbis} asks him to take the hidden mirror that the captain is bound to
have, in violation of Omnian rules, and reflect the sunlight in response.

\subsubsection{\Gls{Gom} tells \Gls{Bru} that the world is flat and carried on the back of
    \Gls{Atuin}}
In his dream, \Gls{Gom} recalls the rise of his religion, starting with the conversion of a shepherd
and the hostile takeover of the popular God of the time. He wonders why he has lost so much power,
in spite of his religion itself doing so well in Omnia. The ship's captain reveals to \Gls{Bru} that
he has seen the edge of the Disc, and reaffirms his belief in the flat world. \Gls{Bru} is alarmed
to learn that the world is not a sphere, as taught by Omnianism, and asks \Gls{Gom} if he had any
part in Creation.

\subsubsection{\Gls{Gom} inspects the welcoming committee at the Ephebian harbour}
\Gls{Vorbis} is amused by \Gls{Bru}'s unwavering belief in the Quisition's mandate to punish anyone
suspected of the slightest heresy. \Gls{Bru} conveys the details of the flashes from the coastline,
and refrains from asking what they mean. As the ship nears the coast of Ephebe, \Gls{Gom} looks at
the soldiers massed in the harbour, and wonders if the Omnians want peace in order to stem their
losses.

\subsubsection{\Gls{Gom} talks about the strategies used by Gods to stay in power}
\Gls{Gom} describes each of the Gods as the diplomats pass by their statues on the road to the
Ephebian palace. \Gls{Bru} is confused by \Gls{Gom} calling the other Gods real, since Omnian
scripture is founded on \Gls{Gom} being the one true God. \Gls{Gom} talks about Gods finding a niche
that would ensure a steady stream of believers, while being obscure enough to avoid any competitors
trying to usurp them.

\subsubsection{\Gls{Bru} and the diplomats reach the Ephebian palace after being led through the
    maze blindfolded}
A naked old man runs down the street to ask a potter for some simple levers and pulleys. \Gls{Gom}
explains that the city tolerates its philosophers for the occasional stroke of military genius that
they are bound to have. Eventually, the Ephebian soldiers blindfold the diplomats and lead them to a
secret meeting yard, lined with Ephebian bowmen. \Gls{Aristo}, secretary to the democratically
elected \Gls{TyrantE} welcomes them, while asking \Gls{Vorbis} not to risk navigating the maze
surrounding the palace on his own.

\subsubsection{\Gls{Gom} asks \Gls{Bru} to find a philosopher in Ephebe}
\Gls{Gom} realizes that \Gls{Bru}'s mind is starting to resemble that of a prophet, and wonders if
he might find an Ephebian philosopher ready to listen to him. He asks \Gls{Bru} to let go of the
rigid rules of fasting, and open up the melons for them to eat. \Gls{Bru} is alarmed at \Gls{Gom}'s
request to head out into the city through the booby-trapped maze, and find a philosopher who might
help him leave his tortoise form.

\subsubsection{\Gls{Bru} tries to ask the barman where he can find a philosopher}
\Gls{Bru} successfully navigates the maze, and finds a bar in the city full of philosophers, about
to erupt into a brawl. \Gls{Xeno} tries to argue that Gods are an outdated system of personifying
unexplained natural phenomena, and is forced to revise his statement when the major Gods all
threaten to smite him in their own ways. \Gls{Bru} then asks the barman for some water, while
believing that alcohol is blasphemous per the teachings of Omnianism.

\subsubsection{\Gls{Vorbis} faces the \Gls{TyrantE}}
The barman suggests \Gls{Didact}, a cheap philosopher who lives by the Library, when \Gls{Bru}
points out his lack of money. The next morning, \Gls{Aristo} leads the diplomats to the ruler's
chamber, with \Gls{Vorbis} reminding \Gls{Bru} that he is to remember every word that is said in
the room. The \Gls{TyrantE} refuses to offer any respect to \Gls{Vorbis}, reminding him that if
their places were switched, the Ephebians would be forced to grovel.

\subsubsection{\Gls{Bru} listens to the \Gls{TyrantE} defend peace treaty}
\Gls{Gom} manages to find the Library, and notices \Gls{Didact} arguing with his nephew \Gls{Urn}
about the slow business leaving them at the edge of starvation. Meanwhile, the \Gls{TyrantE} defends
his citizens stoning Brother \Gls{Murd} to death, when he started to preach in all seriousness at
the City Square. \Gls{Vorbis} is forced to take the draft of the peace treaty and sign it without
any further concessions. \Gls{Bru} realizes that the stories about \Gls{Murd} he had heard from
\Gls{Nhum} might not be true.

\subsubsection{\Gls{Bru} stops \Gls{Didact} from killing \Gls{Gom}}
\Gls{Bru} talks to a slave cleaning his chamber, and learns that their life is better than his own
as a novice priest. \Gls{Bru} asks \Gls{Vorbis} for permission to visit the palace Library, and
is surprised to hear him say yes. He finds \Gls{Urn} and \Gls{Didact} using \Gls{Gom}'s ability to
draw polygons in the sand to earn some money from the philosophers who are ready to place bets on
the next shape. \Gls{Bru} asks them not to eat the tortoise, and tells them that they can make even
more money using its unique talents.

\subsubsection{\Gls{Gom} asks \Gls{Bru} to enquire about books on religion}
\Gls{Gom} asks \Gls{Bru} to lie about his identity, and call himself a clerk looking for information
about the origin of Gods. \Gls{Didact} is convinced by the success of his gambling enterprise, and
takes \Gls{Bru} into the Library to show him the scrolls authored by the finest minds of Ephebe.
\Gls{Bru} finds it fascinating, given the ban on any scientific thought in Omnia, as well as all
art.

\subsubsection{\Gls{Urn} finds a book matching \Gls{Bru}'s request}
\Gls{Urn} fishes out a book on small Gods for \Gls{Bru} to read. \Gls{Didact} works out that
\Gls{Bru} must be lying about being a scribe, and wonders how he was spotted in the tavern after
the diplomats had all been escorted through the palace maze. \Gls{Urn} wonders if \Gls{Bru} flew
over the labyrinth, or arrived late in the night.

\subsubsection{\Gls{Gom} explains how belief in Gods gradually decays once institutions spring up
    to support them}
\Gls{Gom} goes through the book on religion, and tells \Gls{Bru} that the Church of Om has grown
around the central deity and eventually hollowed out the belief in him like termites on a tree.
He asks \Gls{Bru} to take on the burden of being a true prophet, and try to reignite belief in
him, over and above faith in the institutions themselves.

\subsubsection{\Gls{Bru} leads \Gls{Vorbis} to the city gates at night}
\Gls{Bru} asks \Gls{Vorbis} about the truth behind the supposed murder of Brother \Gls{Murd}, and
sees him offer some vague philosophical statement to try and get around lying. \Gls{Vorbis} then
asks him to lead the way through the labyrinth, and murders one of the old guides who comes to
investigate their footsteps. Once outside the maze, \Gls{Vorbis} notices his agent signaling using
a lamp in the mountains outside the city, and asks \Gls{Bru} to lead the way to the gates.

\subsubsection{\Gls{Vorbis} takes over the city using a sneak attack at night}
\Gls{Vorbis} lets his secret force of invaders into the city, where they overwhelm the small force
tasked with defending the palace. \Gls{Bru} tries and fails repeatedly to defy \Gls{Vorbis}'
commands, believing that respect for the chain of command is ingrained too deeply in his mind.
He summons \Gls{Didact}, the author of the pamphlet responsible for triggering the rise of the
\Gls{Atuin} Movement inside Omnia.

\subsection{}
\subsubsection{\Gls{Didact} insults \Gls{Vorbis} on his way out of the throne room}
\Gls{Didact} makes a show of agreeing with \Gls{Vorbis} and his interpretation of the world, and
walks out of the throne room. Just as he leaves, \Gls{Didact} throws his lantern at \Gls{Vorbis}'
face, prompting the Omnian soldiers to hold him in custody. \Gls{Simony} shows up to protect
\Gls{Didact} from the guards sent to hunt him down, and reveals himself to be a member of the
\Gls{Atuin} Movement.

\subsubsection{\Gls{Didact} trusts \Gls{Bru} to save the information in the Library}
\Gls{Simony} arrives at the Library, just as the soldiers sent by \Gls{Vorbis} prepare to torch it.
He sends them away, and asks \Gls{Didact} to come with him to safety, while considering \Gls{Bru}
an untrustworthy member of the Quisition. \Gls{Didact} decides to trust \Gls{Bru} when he mentions
a plan to save the books in the Library, and asks \Gls{Simony} to fetch his pet tortoise from the
palace.

\subsubsection{\Gls{Bru} saves the information in the Library in his mind}
\Gls{Bru} uses his photographic memory to dump the information in the scrolls into his mind, as
\Gls{Urn} and \Gls{Didact} look on speechless. The \Gls{Libra} receives the emergency warning about
the Ephebian Library being burned, and uses his special privileges to portal into the room and
rescue some of the scrolls. A while later, \Gls{Bru} wakes up on the harbour, to see \Gls{Urn}
testing out his steamboat engine.

\subsubsection{\Gls{Simony} asks \Gls{Bru} to provide testimony against \Gls{Vorbis}}
\Gls{Simony} wishes to use \Gls{Bru} as a witness against \Gls{Vorbis}, and asks him to come back to
Omnia to provide testimony. \Gls{Gom} asks to be dropped off in Ankh-Morpork first, with
\Gls{Didact} tagging along, desperate to try and establish a new batch of believers in the big
city. Meanwhile, \Gls{Vorbis} decides to head to the harbour after noticing the absence of any
corpses in the burnt Library.

\subsubsection{\Gls{Gom} saves himself and \Gls{Bru} from the \Gls{QSea}}
\Gls{Bru} starts to feel the knowledge of the scrolls leaking into his mind, even though he merely
memorized the symbols without knowing how to read or write Ephebian. Meanwhile, \Gls{Gom} is fast
asleep, and lets his mind wander the realm of the Gods, where the \Gls{QSea} takes notice of him,
and demands the sacrifice of the steamboat as payment for her favour. \Gls{Gom} reluctantly agrees,
and asks \Gls{Bru} to jump off the steamboat, just as a lightning bolt hits its steam engine.

\subsubsection{The \Gls{QSea} claims every soul on \Gls{Vorbis}'s ship except his own}
The \Gls{QSea} sets her sights on the Omnian battleship closing in on the steamboat, realizing that
it has a full crew of sailors and many more souls for her to claim. The entire crew lose their lives
in the divine hurricane, with \Gls{Vorbis} being the only survivor. \Gls{Bru} and \Gls{Gom} wash
up on a beach, and they head to Omnia, against the wishes of the tortoise.

\subsubsection{\Gls{Gom} decides to follow \Gls{Bru} through the desert}
\Gls{Gom} refuses to accompany \Gls{Bru} to Omnia, when he comes across the bloodied body of
\Gls{Vorbis}, who is miraculously still alive. \Gls{Bru} decides to carry \Gls{Vorbis} back to the
Citadel to hold him accountable for his crimes against \Gls{Murd} and the Ephebians. \Gls{Gom}
crawls through the desert after \Gls{Bru}, and finds him collapsed at the base of a dune, desperate
for water.

\subsubsection{\Gls{Gom} talks about the power of Gods to \Gls{Bru}}
\Gls{Gom} digs into the sand in search of underground water, only to find \Gls{Bru} pour some of it
into the mouth of \Gls{Vorbis}. He tells \Gls{Gom} that Ephebe would stone them to death if they
returned, and is reminded of the discontent simmering within Omnia regarding the stranglehold of
the Quisition. \Gls{Gom} explains the power wielded by the Great Gods in Cori Celesti, and talks
about how they masquerade in different forms depending on the local belief systems of the Disc.

\subsubsection{\Gls{Urn} sets sail for Omnia in the ruins of his steamboat}
\Gls{Urn} talks about sailing to Omnia, now left with nothing but fragments of his steam turbine,
as \Gls{Simony} talks glowingly about the secret \Gls{Atuin} Movement growing within the kingdom.
\Gls{Didact} decides not to worry about the possible death of \Gls{Bru}, who held the knowledge of
the Ephebian Library in his head.

\subsubsection{\Gls{Bru} finds refuge in a small cave, where he kills a snake for food}
\Gls{Bru} kills a snake in a small seaside cave, and starts to take it apart for food. \Gls{Gom}
tries to argue that the snake would have poisoned him if he had shown mercy, and taunts his
developing sense of ethics. Meanwhile, \Gls{Simony} spots some landmarks that point to a village on
the outskirts of Omnia, and tells \Gls{Didact} that he lost his entire homeland to an invasion by
\Gls{Vorbis}.

\subsubsection{\Gls{Gom} decides to find the desert lions in search of a water source}
\Gls{Gom} fights off the spirits of forgotten gods that come to the cave, seeking the mind of
the two humans inside it for conversion. He then starts to think about finding water, and decides to
track down the desert lions. The next day, \Gls{Gom} asks \Gls{Bru} to bring \Gls{Vorbis} along,
while remaining vague about how he has tracked down their next water source.

\subsubsection{\Gls{Bru} decides to help the hungry lion, when he sees it struggling to stay alive}
\Gls{Bru} carries \Gls{Vorbis} to the next seaside island, which shows signs of vegetation. As an
injured lion crawls up to them, with a spear sticking out of its side, \Gls{Bru} decides to help it.
\Gls{Gom}, who is unapologetic about leaving \Gls{Vorbis} for the lion to feed on, hides in his
shell, bemoaning the idealism of his only believer. \Gls{Bru} pulls out the spear, and tends to the
open wound, as \Gls{Gom} notices the stairs carved into the lion's cave.

\subsubsection{\Gls{Gom} remains evasive about his duty towards the people of Omnia}
\Gls{Gom} tries to argue that \Gls{Vorbis} deserves to be left for dead, but \Gls{Bru} sees the
torturer stare off into the distance, having uttered no words since he was rescued. \Gls{Bru}
decides to reject \Gls{Gom}'s attempt to disavow the actions of the Church, and reminds him that he
seeded the faith the people of Omnia, and is therefore responsible for their well-being.

\subsubsection{\Gls{Didact} fails to give the revolutionaries the fiery speech expected by
    \Gls{Simony}}
\Gls{Didact} begins his speech to the believers of the \Gls{Atuin} Movement, gathered by
\Gls{Simony} and his agents in the village. \Gls{Urn} is busy helping the blacksmith build a
steam-powered battering ram to break through the Citadel's main gates, as \Gls{Simony} thinks about
the need for a symbol to inspire the people's revolution.

\subsubsection{\Gls{Bru} sees the orphaned Gods trying to get through to him}
\Gls{Gom} fends off the orphaned Gods once again, as \Gls{Bru} and \Gls{Vorbis} lie down to sleep
at night. \Gls{Bru} sees visions of a grand feast, before \Gls{Gom} banishes the orphaned Gods,
and learns that they are trying to appeal to him by showing him visions of pleasure. A while later,
they run into \Gls{Ungul}, a recluse that is friendly to all the orphaned Gods.

\subsubsection{\Gls{Bru} talks to \Gls{Ungul} about water sources in the deep desert}
\Gls{Bru} quickly figures out that \Gls{Ungul} has gone mad, under the influence of a Small God
named \Gls{Angus}, who is only visible to him. \Gls{Bru} decides to leave as soon as \Gls{Ungul}
tells him the secret of extracting water from the desert cacti, and leaves the madman to his
negotiations with the orphaned Gods. Meanwhile, Ephebe is freed by a citizen revolt, with the
Omnian occupying force scattered in the absence of \Gls{Vorbis}.

\subsubsection{\Gls{Vorbis} suddenly comes to his senses at the edge of the Omnia}
\Gls{Gom} tries to eat some leaves in a thorny shrub at the edge of the Omnia, and sees
\Gls{Vorbis} suddenly sit up at attention. He realizes that one of the stronger Small Gods has
taken over his mind, and watches him march off to the village, with \Gls{Bru} slung over his
shoulders.

\subsection{}
\subsubsection{\Gls{Bru} wakes up at the Citadel and learns that \Gls{Vorbis} has twisted the truth}
\Gls{Ungul} is saved by the old lion, when \Gls{Angua} uses all of its meager strength to throw
a rock at it, hitting it mid-leap. Meanwhile, \Gls{Bru} dreams of the black desert which is the
Omnian trial to enter the afterlife, and wakes up in \Gls{Nhum}'s chamber. \Gls{Nhum} tells him
that \Gls{Vorbis} arrived at the Citadel, with a donkey bearing him on its back. \Gls{Nhum} is
excited to welcome \Gls{Vorbis} as the next prophet, and speculates that \Gls{Bru} will also get a
promotion for accompanying him through the hostile desert.

\subsubsection{\Gls{Bru} listens to \Gls{Vorbis} declare himself the next Prophet}
\Gls{Bru} goes to the private garden of the higher bishops, and finds \Gls{Vorbis} there, arguing
for the elevation of his faithful follower to the rank of Archbishop. \Gls{Bru} realizes that
\Gls{Vorbis} is once again falsifying the events of the last few days, in order to push the
narrative that will lead to his advancement to the rank of Prophet.

\subsubsection{\Gls{Bru} wonders why \Gls{Vorbis} is continuing to keep him around}
\Gls{Bru} wonders why \Gls{Vorbis} is afraid of him, given how quick he is to silence dissenters.
He is taken to an underground workshop, where a new torture implement is being designed, in the
shape of a large metallic turtle to which the victim might be chained, and eventually killed by
a furnace heating the metal from underneath. On his way back to his dormitory at the novices' hall,
\Gls{Bru} runs into \Gls{Dhblah}, who already knows about his impending promotion.

\subsubsection{\Gls{Bru} returns to his vegetable garden at the Citadel}
\Gls{Bru} impulsively goes to the vegetable garden, and starts tending to the plants, as \Gls{Sweep}
watches silently from his nearby tent. He wonders why \Gls{Vorbis} is playing the carrot and stick
game with him, and realizes that he needs a supporting testimony to reaffirm his position as
Prophet.

\subsubsection{\Gls{Urn} works out some of the issues with his battering ram}
\Gls{Urn} describes the problems with his steam-powered battering ram to \Gls{Simony}, and
Sergeant \Gls{Fergm}, who is a palace guard sympathetic to their cause, talks about the hydraulic
system being used behind the scenes to operate the Main Gate. \Gls{Urn} realizes he only needs his
machine to last long enough for one ramming against the Gate, and hopes for the best.

\subsubsection{\Gls{Vorbis} decides to let the battering ram interrupt his coronation}
\Gls{Vorbis} receives news of the battering ram from one of the village blacksmiths assisting in
its construction. He orders the man's father to be released from prison, and asks Deacon \Gls{Cusp},
one of his junior inquisitors, to increase the security for his ceremonial ascension to Prophet.
\Gls{Sweep} sabotages \Gls{Urn}'s steel-making process, and comes back to the Citadel's vegetable
garden to find \Gls{Bru} despairing over the loss of \Gls{Gom}. He reminds \Gls{Bru} that he can
simply make up some commandments just like all the Prophets that came before him.

\subsubsection{\Gls{Gom} lands in some water a fair distance away from the Citadel}
\Gls{Gom} crawls to the palace, having been dropped by an eagle once again into a ditch filled with
water. He is able to sense the direction of \Gls{Bru}'s mind, while lacking the energy to
telepathically contact the young boy. Meanwhile, \Gls{Sweep} asks \Gls{Cusp} to move aside as he
inspects the Grand Hall where \Gls{Vorbis} is to be coronated.

\subsubsection{\Gls{Bru} is greeted by \Gls{Urn}, who is intent on sabotaging the Main Gate}
\Gls{Bru} watches \Gls{Urn} and \Gls{Fergm} sneak into the Citadel through an underground tunnel.
\Gls{Urn} greets him, and is surprised that he is still alive. He asks \Gls{Bru} to stay
away from the Citadel gates, and heads on his secret mission. \Gls{Fergm} then leads him to the
chamber with the hydraulic controls for the Main Gate, and points out the man-powered treadmills
that power the mechanism. \Gls{Urn} decides to wait for the horn that will signal the battering ram
breaking through the Gate.

\subsubsection{\Gls{Urn} and \Gls{Fergm} successfully open the Main Gate}
\Gls{Bru} prepares to confront \Gls{Vorbis}, even as the preliminary ceremony is in progress at the
Great Hall of the Citadel. \Gls{Gom} senses his intentions and is dismayed by the realization that
he is acting too soon, and without any acolytes to provide backup. \Gls{Simony} pulls the lever to
start the battering ram, and it breaks because of \Gls{Sweep}'s meddling. \Gls{Cusp} discovers
\Gls{Urn} meddling with the hydraulics, and is pushed by \Gls{Fergm} into the pit holding the
counterweights.

\subsubsection{\Gls{Bru} is condemned to death by \Gls{Vorbis}}
\Gls{Bru} confronts \Gls{Vorbis} inside the Temple, as the guards surround him. He finds himself
unable to punch \Gls{Vorbis}, and thinks about how futile it would be to try and offer his own
version of the journey through the desert. \Gls{Vorbis} is incensed by \Gls{Bru}'s self-restraint,
and orders him to be tortured to death. \Gls{Gom} manages to get grabbed by an eagle, and hold its
life hostage in exchange for an express flight to the Citadel.

\subsubsection{\Gls{Urn} asks \Gls{Simony} to go ahead with the coup, in spite of the odds}
\Gls{Urn} is sad to see \Gls{Simony} at the entrance to the Temple, hesitating to trigger the coup.
He tells \Gls{Simony} that \Gls{Vorbis} has succeeded in driving the idealism out of his mind, and
points out the martyr that \Gls{Bru} is going to become, roasting on the iron turtle in front of
the guests gathered for the coronation.

\subsubsection{\Gls{Gom} lands on \Gls{Vorbis}'s head, killing him}
\Gls{Bru} senses \Gls{Gom}'s voice in his mind, and looks up at the sky to see the faint speck of
the eagle flying towards the Temple. He tells the audience that \Gls{Gom} will soon judge between
the two of them, and says sorry to \Gls{Vorbis}, just as \Gls{Gom} lands on him at great speed,
instantly crushing his skull and killing him.

\subsubsection{\Gls{Gom} regains his divine form, as the gathered clergy start to believe in him}
\Gls{Gom} grows in power, and is restored to his godly form, when the gathered audience renew their
belief in him. \Gls{Bru} asks him to base his religion on rules that he will have to obey as well,
and makes a pact to revise the commandments every hundred years. \Gls{Simony} learns that all the
neighbouring kingdoms have banded together to eradicate Omnia, and asks \Gls{Gom} to defend the
kingdom.

\subsubsection{\Gls{Bru} wishes to negotiate a peace treaty with the invading alliance}
\Gls{Bru} decides to march into the desert, in order to parley with the invading armies. He forbids
any military force from following along, and \Gls{Simony} asks the others to form a peace delegation
just in case. \Gls{Dhblah} asks \Gls{Gom} for some business opportunities, and is given leave to
sell tortoise memorabilia to his heart's content. \Gls{Gom} then vanishes, swearing never to appear
in his divine form ever again.

\subsubsection{\Gls{Bru} tries to negotiate with \Gls{Agrav}}
\Gls{Agrav}, the Ephebian general in charge of the invasion, notices the deserted beach in Omnia,
and wonders if the enemy has advance knowledge of their attack. \Gls{Bru} tells the invaders that
he intends to surrender as the presumptive head of state, and swears to open the country up to
foreign religions. \Gls{Gom} is hoodwinked into playing along, when \Gls{Bru} points out that the
other Gods are powerless entities, which will only serve to highlight his own value.

\subsubsection{\Gls{Urn}'s efforts at peace are ruined by \Gls{Simony} assembling his army on the
    beach}
\Gls{Urn} is horrified to learn that his steam engine is going to be used against the Ephebian
invaders, as \Gls{Simony} lines up his forces at the beach to defend Omnia. \Gls{Bru} punches him
when he starts to talk about honour and glory, while fondly recalling \Gls{Ungul}'s solitary life
in the desert. \Gls{Bru} then walks away to the edge of the battlefield, where \Gls{Didact} laughs
at his child-like idea of ending the war before it starts.

\subsubsection{\Gls{Gom} puts an end to the divine meddling in the invasion of Omnia}
\Gls{Gom} goes to Cori Celesti, and finds the major Gods playing a game with the Disc in their
secret hideout of Dunmanifestin. He realizes he has lost his callous disregard for human life, from
spending too long among the mortals, and causes a ruckus to delay the battle. \Gls{Gom} blackmails
the Major Gods into stopping the lightning-storm assaulting the invading ships, and makes them
deliver a message of self-reliance to the gathered armies.

\subsubsection{\Gls{Bru} reforms Omnia, and gets rid of the secret police}
\Gls{Bru} starts to think about reforming Omnia, starting with the appointment of \Gls{Didact} to
the head of the clergy. He also decides to shut down the Quisition, and appoints \Gls{Simony} as
in charge of law enforcement. A hundred years later, \Gls{Bru} dies, and finds \Gls{Vorbis} waiting
for him in the desert of the afterlife. \Gls{Death} is impressed by \Gls{Bru}'s compassion, as he
picks \Gls{Vorbis} up to help him reach the afterlife.