\section{Pyramids}


\subsection{}
\subsubsection{\Gls{Teppic} takes part in the Assassins' Guild examination}
An old priest \Gls{Dios} wakes up from his stone bed in a small chamber, and comes out at sunrise
to guide the world with his voice. At sunset, the assassin \Gls{Teppic} kits himself out with a
variety of lethal tools, and departs for his exam, while wondering who the examiner is going to be.
As he reaches \Gls{Mericet}'s quarters, intent on killing him to earn an automatic passing grade,
he works out that the body on the bed is a decoy, and calls out for the master.

\subsubsection{\Gls{Teppic} loses his footing when trying to jump across an alley}
\Gls{Teppic} answers \Gls{Mericet}'s questions with ease, and starts the parkour portion of the
exam. He misses a jump, and his life flashes before him, starting with his father \Gls{KingTep},
the King of Djelibeybi, sending him to Ankh-Morpork to get a high-quality education. He had
received some empty advice from his father, and sacrificed a chicken to \Gls{Khuft}, the empire's
mythical founder, before setting off for the city.

\subsubsection{\Gls{Teppic} recalls the start of his Assassin training}
\Gls{Teppic} recalls his first day in the Assassins' Guild, surrounded by other teenagers from
wealthy families, with their trunks full of clothing. He manages to grab a small handhold in the
building's wall to break his fall, as he recalls his first research assignment, that led him to the
Guild Library to look at common ingress routes into palaces. In the present, he manages to use two
of his daggers to carve out footholds for himself on the crumbling wall.

\subsubsection{\Gls{Teppic} manages to break into the building using a window}
\Gls{Teppic} talks to his friend \Gls{Chidder} about the bankruptcy facing his kingdom, with all
the money going into building and furnishing the pyramids. \Gls{Chidder} is amused by the idea that
the prince has been forced to go out and earn money in order to run the kingdom, as they see
a young boy being comforted by his mother. In the present, \Gls{Teppic} manages to break into the
window that had caught his fall, and evades \Gls{Mericet}, who is waiting for him at the trapdoor
leading to the roof.

\subsubsection{\Gls{Teppic} recalls the religiosity among his fellow students at the Guild School}
In the past, \Gls{Teppic} remembers the first night in the spartan dormitory, where \Gls{CWright}
had stopped \Gls{Ludorum} from sacrificing a goat in the ritual circle around his bed, and this
ruining their sleep. \Gls{Chidder} gets into a fight with \Gls{CWright}, and \Gls{Teppic} uses the
opportunity to assuage \Gls{Ludorum}'s fears of divine punishment. In the present, \Gls{Teppic}
continues his journey across the rooftops, mindful of the booby-traps set for careless students,
designed to cripple or even kill them.

\subsubsection{\Gls{Teppic} finishes his test by accidentally shooting the target}
\Gls{Teppic} reaches the room where his designated target is, and finds \Gls{Mericet} there waiting
for him, somehow having beat him to the spot. He navigates all the traps in the corridor, and faces
the moment of delivering the killing blow. He hesitates, and decides to intentionally fire at the
window, only for the crossbow bolt to ricochet into the target. \Gls{Mericet} gives him his
certificate with a passing grade, and he anxiously pulls back the blanket on the sleeping figure.

\subsubsection{\Gls{Teppic} and some of his classmates pass their exams}
\Gls{Teppic} returns to the Guild to find \Gls{Chidder} waiting for him, having already passed
a much easier run. \Gls{Teppic} recalls the teachings of \Gls{Cruces} about why they kill, as
\Gls{Ludorum}, the son of a famous assassin on a full scholarship, also returns from his exam.
They conclude that \Gls{CWright} isn't coming back after waiting for a bit, and proceed to their
celebratory meal.

\subsubsection{\Gls{KingTep} has a bad dream about the sun no longer rising the next day}
\Gls{KingTep} has a nightmare about the sun failing to rise, and the people blaming him for failing
in his duties as pharoah and God-King. He takes off from the palace and flies like a bird in
his dream. Meanwhile, \Gls{Teppic} and his drunk friends are held up by some unlicensed theives, who
plan to kill them just for sport.

\subsubsection{\Gls{KingTep} dies when he sleepwalks off a high wall}
\Gls{KingTep} dies in a sleepwalking incident, and is told by \Gls{Death} that he will have to hang
around for the traditional burial rites to be completed. He starts to become disillusioned about
the embalming and pyramid business, just as the spirit of the god inside him travels to \Gls{Teppic}
through a seagull, that perches next to the three assassins on the bridge.

\subsubsection{\Gls{Dios} tries to maintain order among the clergy after the death of \Gls{KingTep}}
First Minister \Gls{Dios} oversees the embalming of the dead King, with \Gls{Dil} in charge of
performing the rites. \Gls{Koomi}, a fellow priest intent on ascending to his position, talks about
the commissioning of the new pyramid and the recalling of prince \Gls{Teppic}. \Gls{Dios} feels his
age catching up to him, and decides to visit the necropolis again, while wondering how easily
\Gls{Teppic} will fit into his new role as God-King.

\subsubsection{\Gls{Ludorum} and \Gls{Chidder} try to drag \Gls{Teppic} to safety}
\Gls{Ludorum} and \Gls{Chidder} hastily drag \Gls{Teppic} back to the Guild, wondering if his
newfound powers are responsible for the flooding of the river Ankh, as well as the rapid growth of
greenery on the streets they pass through. At the local hospital, the doctor declares \Gls{Teppic}
brain-dead, and tells the two of them that he cannot help the boy. \Gls{Teppic} suddenly comes to
his senses, and asks to be taken home.

\subsubsection{\Gls{Teppic} gains divine knowledge as a result of the God-King's spirit entering
    him}
\Gls{Teppic} rattles off the details of \Gls{Chidder}'s family smuggling operation, to his complete
shock. He asks to borrow their fast boat so that he can get back to the palace in Djelibeybi, and
wonders if \Gls{KingTep} is indeed dead. Meanwhile, \Gls{KingTep}'s spirit watches on with interest,
as his body is prepared by the master embalmer \Gls{Dil} and his apprentice \Gls{Gern}. He wonders
why he placed so much store by these meaningless rituals when he was alive, as he observes his
organs being pickled before the corpse is dumped in preservative.

\subsubsection{\Gls{Teppic} prepares for the coronation ceremony}
\Gls{Teppic} arrives at the palace using \Gls{Ludorum}'s smuggling boat under cover of darkness,
and is happy to see the huge statue of \Gls{Khuft} in front of the palace. \Gls{Dios} fusses over
the precise arrangements of relics in \Gls{Teppic}'s hands, as he prepares for his first public
appearance as God-King. \Gls{Teppic} tries to talk about the technological advancements he saw in
Ankh-Morpork, such as plumbing, but \Gls{Dios} remains steadfast regarding the value of tradition.

\subsection{}
\subsubsection{\Gls{KingTep} observes the end of the embalming process on his corpse}
\Gls{Teppic} is shocked at \Gls{Dios} suggesting the new King marry his aunt, while lamenting the
absence of any sisters who might better suit the role. \Gls{Dios} talks about familial marriage
keeping the divine blood pure, and stresses on the need to follow tradition. Meanwhile,
\Gls{KingTep} watches his corpse get stuffed with hay and wrapped in bandages, dismayed at the fact
that the embalmers cannot sense him beyond feeling the faintest breeze.

\subsubsection{\Gls{Dios} takes \Gls{Teppic} to visit his embalmed father}
\Gls{Dios} leads the royal procession for \Gls{Teppic} to pay his last respects to his father, while
brushing off the fact that he had been Head Priest for at least a hundred years. \Gls{Teppic} learns
about some of the mythical Djel emperors from a time when the empire spanned a much larger area,
as \Gls{Dios} laments his inability to read hieroglyphics.

\subsubsection{\Gls{Teppic} is egged on by \Gls{Dios} to announce the building of a pyramid for
    \Gls{KingTep}}
\Gls{KingTep} desperately tries to ask \Gls{Teppic} not to build a pyramid, so that he can explore
the world in his spirit form for a while longer. \Gls{Teppic} is egged on by \Gls{Dios}, who claims
to be the conduit to the dead, into commissioning a grand pyramid for the deceased King, even as
\Gls{Teppic} has second thoughts about it.

\subsubsection{\Gls{Dios} asks \Gls{Ptacl} for a deluxe pyramid}
At the office of the chief architect \Gls{Ptacl}, \Gls{Teppic} tries to endorse his father's desire
to be buried at sea, only for \Gls{Dios} to overrule him using the weight of tradition. The priest
demands the most elaborate pyramid possible, including all the bells and whistles, as \Gls{Teppic}
hopelessly plays along, too afraid to anger \Gls{Dios} any further.

\subsubsection{\Gls{Ptacl} promises a magnificent pyramid for \Gls{KingTep}}
\Gls{Ptacl} adds on every feature he can think of, hoping to extract as much of a profit as possible
from the work order. \Gls{Dios} reminds him that the pyramid will still need to be ready in three
months, in time for the Djel river's flood season. \Gls{Dios} suppresses some of his aging symptoms
again, prompting \Gls{Teppic} to ask him if he is alright.

\subsubsection{\Gls{Dios} goes to his chamber in the necropolis to prolong his life}
\Gls{Teppic} takes stock of the network of pyramids lining the fertile riverbank, stretching all
the way from the delta to the mountain ranges that protect the kingdom from desertification.
\Gls{Dios} discreetly makes his way to a jetty on the Djel river, ready to cross it to head to his
private chamber in the necropolis and indulge in his life-lengthening ritual.

\subsubsection{\Gls{Ptacl} is forced to think of the financing of the \Gls{KingTep}'s pyramid}
\Gls{Ptacl} tries to impress upon his twin sons the importance of completing the pyramid contracts
for the royal family, regardless of being paid. The sons argue about the finances of building such
an impressive pyramid, against the glory of being the first architects to achieve the feat.
\Gls{Ptacl} thinks about the profits his firm has accumulated from the small tomb orders placed
by commoners, and of how the royal pyramids are a loss-making endeavour only serving as a marketing
gimmick.

\subsubsection{\Gls{Teppic} decides to bring some of the comforts of the city back to the palace}
\Gls{Teppic} longs for the soft mattresses, and plumbing of Ankh-Morpork, deciding to have some of
these technologies imported to his palace. He has bizarre dreams at night, featuring his ancestor
spirits berating him for abandoning tradition, and his father's voice almost imperceptibly asking
him not to be buried under a pyramid. The next morning, \Gls{Dios} returns to the palace with a
spring in his step, as \Gls{KingTep}'s spirit listens to \Gls{Dil} talk about subtle alterations to
the King's ugly visage.

\subsubsection{\Gls{Ptacl} answers \Gls{Teppic}'s questions about the pyramid}
\Gls{Ptacl} dreads having to interact with the new King, but plays along when \Gls{Dios} subtly
warns him not to protest. \Gls{Teppic} is impressed by the newfound levitation spells, of which
\Gls{IIb} is a master. The pyramid slowly takes shape, as each limestone block is hauled into
position after being rendered temporarily weightless. \Gls{Teppic} is horrified to learn that a
worker who he shook hands with will have to have the limb amputated, lest he defile it in the course
of everyday living.

\subsubsection{\Gls{Ptacl} learns about the temporal anomalies in the Great Pyramid}
\Gls{Dios} tells \Gls{Teppic} that the delivery of the goods from Ankh-Morpork is likely beset by
Klatchian pirates at the delta, fully aware that the King knows of his foul play. \Gls{Ptacl} is
called to the pyramid work-site by one of his foremen, who complains about a time-dilation node
having ruined his lunch. \Gls{IIb} wonders if the nodes forming before the pyramid is finished and
capped, is a result of the gigantic scale or some other strange phenomenon.

\subsubsection{\Gls{Ptacl} thinks about using the time loops in the pyramid to his advantage}
\Gls{Ptacl} starts to think about using the time dilation to get more work done inside the pyramid,
and asks \Gls{IIb} to work out where the time loops would manifest themselves. Meanwhile, \Gls{Dios}
warns \Gls{Teppic} about the need to play the two neighbouring empires of Tsort and Ephebe against
each other, in order to maintain Djelibeybi's status as a neutral third party separating the two.

\subsubsection{\Gls{Teppic} learns that the priest class takes care of the diplomacy}
\Gls{Teppic} decides to learn about the Ephebian science considered heresy by \Gls{Dios}, as the
delegation from Tsort and Ephebe come to greet him. \Gls{Dios} tells him that the actual diplomacy
has already been carried out behind closed doors by the high priests, and that the King is only
addressing the diplomats as a matter of tradition.

\subsubsection{\Gls{Teppic} faces \Gls{Ptraci} after she refuses to be entombed with \Gls{KingTep}}
\Gls{Teppic} is annoyed to see his judgements being reinterpreted by \Gls{Dios} to favour the
priests, as he dispenses justice in his throne room. After some cases are heard, \Gls{Ptraci} is
brought by some guards to the throne room, with \Gls{Dios} accusing her of refusing to be entombed
with \Gls{KingTep}, who favoured her above all the other handmaidens.

\subsubsection{\Gls{Teppic} rescues \Gls{Ptraci} from the palace prison}
\Gls{Dios} declares that \Gls{Ptraci} is to be cast into the river for the crocodiles to feed, as
\Gls{Teppic} decrees that she be allowed to walk free. \Gls{Teppic} dones his assassin hood at night
and breaks into the prison to rescue \Gls{Ptraci}. He is alarmed to find a prisoner in the next room
yell for the guards when he suggests running away.

\subsubsection{\Gls{Ptraci} is convinced to let go of the fear of divine retribution}
\Gls{Ptraci} points out the fact that she comes from a long line of handmaidens, leading
\Gls{Teppic} to recognize the resemblance she has to one of the books he saw in the Guild Library.
He asks \Gls{Ptraci} to hide in one of the discarded sarcophagi in the sculptor's workshop, and
promises to come back the next day with food.

\subsubsection{\Gls{Dios} decides to track down the escaped prisoner and the rescuer}
\Gls{Dios} comes into \Gls{Teppic}'s chamber the next morning to inform him of the prison break,
and orders a thorough search of every room in the palace. \Gls{Teppic} wishes to visit his father's
embalming chamber to slip some food to \Gls{Ptraci}, but is dismayed to see \Gls{Dios} tail him.
\Gls{Dios} has his guards open the casket in which \Gls{Ptraci} was hidden, only to find it empty.
He frantically looks into the other caskets, as \Gls{Teppic} realizes she must have left the palace
at night.

\subsubsection{\Gls{Dios} tries to interrogate \Gls{Ptacl} at the Great Pyramid work-site about the
    missing prisoner}
\Gls{Dios} talks to \Gls{Ptacl} about the impressive progress on the Great Pyramid, as \Gls{Teppic}
wanders off to interact with the workers. \Gls{Ptacl} learns from the various clones of his sons,
that every worker has used the time loops to duplicate themselves twenty times, leading to an
impossible wage bill in the near future. Against the advice of his sons, \Gls{Ptacl} puts off the
capping of the pyramid to the next morning, when an elaborate ceremony has been set up to attract
potential customers.

\subsubsection{\Gls{Teppic} decides to help \Gls{Ptraci} escape the palace at night}
\Gls{Teppic} is annoyed at the number of soldiers guarding his chamber, per \Gls{Dios}' concerns
about the King's safety. He sneaks out of his room at night, and goes to the embalming chamber to
help \Gls{Ptraci} out of the sarcophagus. The ghost of \Gls{KingTep} tries to tell him that
\Gls{Ptraci} is his half-sister, and realizes he cannot be heard by the living.

\subsubsection{\Gls{IIa} and \Gls{IIb} decide to cap the Great Pyramid at night}
The turbulence of a gathering storm causes the magical capstones of the pyramids to become unstable,
with \Gls{IIb} dismayed by the idea of delaying the capping ritual on the Great Pyramid. Two of the
clones of \Gls{IIa} and \Gls{IIb} decide to take the capstone up the scaffolding to the top of the
Great Pyramid by themselves.

\subsubsection{\Gls{Dios} asks \Gls{Teppic} to leave the kingdom and never return}
\Gls{Teppic} and \Gls{Ptraci} are caught by some guards hidden in the stables, who then summon
\Gls{Dios}. \Gls{Dios} refuses to acknowledge \Gls{Teppic}, and instead declares him the killer of
their rightful king, who is to be fed to the crocodiles post-haste. \Gls{Teppic} manages to slip
into the shadows in the confusion, as \Gls{Ptraci} is apprehended. \Gls{Dios} insists on wanting
\Gls{Teppic} gone from the kingdom, indirectly letting him know that he is aware of the assassin
being the King in disguise.

\subsubsection{\Gls{Ptraci} and \Gls{Teppic} escape the palace using the spacetime instability}
\Gls{IIa} and \Gls{IIb} reach the top of the pyramid, and hear loud creaking coming from the inside.
The time stored by the other pyramids to be sent up into the sky is instead diverted to the Great
Pyramid's top as a result of its sheer size. A large spacetime instability spreads out from the
Great Pyramid, giving \Gls{Teppic} the opportunity to snatch \Gls{Ptraci} from the hands of
\Gls{Dios}, and flee the palace on the back of the lone camel left in the stables.

\subsubsection{\Gls{YouB} works out the danger posed by the spacetime instability}
The camel, who is named \Gls{YouB}, and is the smartest mathematician on the Disc, spits smartly
in the face of \Gls{Dios}, and runs off as the guards look on in horror, afraid to pre-empt the
High Priest. \Gls{Teppic} looks at the Great Pyramid across the river, and notices it shifting
between the usual four-sided shape and a forbidding eight-sided shape in the blink of an eye.

\subsubsection{\Gls{Teppic} finds the entire kingdom of Djelibeybi missing in the valley}
The Great Pyramid lets loose all of the time stored up in its stone foundations at once, just as
\Gls{YouB} scales the valley-side and reaches the start of the surrounding desert. Once a great
shockwave of silence washes past them, \Gls{Teppic} looks back at the valley, and finds that the
kingdom has vanished.

\subsubsection{\Gls{Dil} and \Gls{Gern} notice the old Gods coming to life in the skies}
\Gls{Dil} and \Gls{Gern} rush to the palace, hoping to find some answers for the celestial woman
they see holding up the stars in the sky, and the giant beetle rolling the sun across the horizon
at dawn. \Gls{Dil} realizes that the world has somehow changed to be exactly what the ancient Djel
legends describe.

\subsection{}
\subsubsection{\Gls{Teppic} and \Gls{Ptraci} try to survive in the heat of the desert}
\Gls{Teppic} manages to catch a small glimpse of the alternate dimension in which the kingdom
of Djelibeybi has been confined, before the real-time wasteland comes back again. \Gls{Ptraci}
continues to disbelieve him when he tries to explain that he is the King, and asks if his power to
flood the Djel river can conjure some water in the hot desert sun.

\subsubsection{\Gls{Ptraci} and \Gls{Teppic} seek help in Ephebe}
\Gls{Ptacl} finds \Gls{IIb} in the wreckage of the workers' tent with minor bruises, and drags him
to see \Gls{IIa}, whose body has been twisted by the dimensional instability. \Gls{Teppic} uses the
history lessons from the Guild School to explain the primitive system of democracy that had evolved
in Ephebe, where he intends to go for help in restoring his kingdom.

\subsubsection{\Gls{Ptraci} and \Gls{Teppic} make it to a desert outpost where two Ephebians are
    arguing about tortoises}
\Gls{Ptacl} refuses to listen to \Gls{IIb}'s explanations about his twin brother \Gls{IIa} becoming
a two-dimensional entity, whose third dimension has become space. \Gls{Teppic} gets \Gls{Ptraci}
to hide behind a sand dune, and approaches two Ephebian philosophers, busy arguing about the
postulate that an arrow cannot hit a live tortoise.

\subsubsection{\Gls{Koomi} senses an opportunity to rid himself of \Gls{Dios} and his loyalists}
The two Ephebian philosophers - \Gls{Xeno} and \Gls{Ibid} - pack up their experimental tortoises
for the day, as \Gls{Ptraci} arrives in response to \Gls{Teppic}'s signal of safety. With
\Gls{Dios} left in a catatonic state, the other priests try to make sense of the various gods
fighting for dominance in the sky. \Gls{Koomi} quickly asks for one of the priests who expresses
atheist sentiments to be thrown to the crocodiles in the river.

\subsubsection{\Gls{Koomi} decides not to defy \Gls{Dios} openly for the time being}
\Gls{Dios} throws the King's golden mask on the floor, and tells the other priests that the Old
Gods have no business interfering with the kingdom. \Gls{Koomi} tries to remind him that they are
the foundational basis for all of Djel culture, and quickly realizes he has no backers among the
other priests. \Gls{Dios} is surprised to see \Gls{Koomi} play nice, and is utterly incapable of
dealing with the crisis in the heavens, since all of his mental faculties have been spent
reinforcing long-held traditions.

\subsubsection{\Gls{Teppic} describes the vanishing of his kingdom to the philosophers}
\Gls{Xeno} and \Gls{Ibid} take the newcomers to a nearby inn, where they listen to the predicament
with the vanishing empire of Djelibeybi, and suggest seeking out \Gls{Pthag} for help with
geometric calculations. Meanwhile, in the pocket dimensions to which contains the Old Kingdom,
\Gls{KingTep} takes his first steps as a zombie, hoping to recover his organs which are still in
the many jars surrounding his embalmed body.

\subsubsection{\Gls{KingTep} learns about recent events from \Gls{Dil}}
\Gls{KingTep} tries to put \Gls{Dil} at ease by complimenting him on his embalming work, and is
glad to find them accept the animated mummy in front of them as another effect of the anomaly.
\Gls{Dil} talks about soldiers trying to find help in the neighbouring kingdoms, and finding the
routes leading back to the Old Kingdom, as if trapped in an infinite loop. He complains about the
Gods now wreaking havoc in the sky, and hopes to return to the real world, where they are figments
of imagination.

\subsubsection{\Gls{Teppic} listens to \Gls{Pthag}'s musings about pyramids}
At the symposium, \Gls{Teppic} talks to the philosopher close to him who does not seem to be
interested in the loud argument occupying the others gathered in the hall. He introduces himself as
the Listener \Gls{Endos}, causing \Gls{Teppic} to quickly lose interest and look for \Gls{Pthag}.
He finds the geometrician fussing about the value of pi, and mumbling something about pyramids
sucking up the time in Djelibeybi, causing the kingdom to remain trapped in the past.

\subsubsection{\Gls{Teppic} learns about the nature of pyramid magic from \Gls{Pthag}}
\Gls{Pthag} then talks about the spacetime in the Old Kingdom being jumbled, causing it to vanish
into a pocket dimension so as to not upset the balance of the Disc. \Gls{Ibid} wishes to halt the
Ephebian invasion of Tsort, now that there is no buffer kingdom in between. \Gls{Ptraci} wishes to
get on one of the ships at the Ephebian docks, and explore the world, with no regard for the Old
Kingdom.

\subsubsection{\Gls{Teppic} is unsure about saving the Old Kingdom from its banishment}
\Gls{Teppic} is in two minds about going back to the Old Kingdom and trying to rescue it, as
\Gls{Ptraci} takes him to the harbour with the grand lighthouse. A while later, \Gls{Chidder} runs
into them clad in the richest silks and jewelry, with his boat parked nearby. He invites them onto
his small ship, clearly a pirate vessel remodeled into a wealthy merchant ship for the purposes of
hiding in plain sight.

\subsubsection{\Gls{Ptacl} is persuaded to let go of his obsession with making pyramids}
\Gls{Ptacl} slowly comes to the conclusion that pyramids are bad for business, and swears to listen
to his son \Gls{IIb} about focusing on other architectural projects. He spots the mummified
\Gls{KingTep} coming to the necropolis with two humans in tow, all bearing hammers.

\subsubsection{\Gls{Chidder} asks \Gls{Teppic} to join his family business of sea trade}
\Gls{Chidder} reassures \Gls{Teppic} that he is not a pirate, and is only using the tools on the
ship to fight off pirates who attack his operations. He starts to suggest moving his family company
to Djelibeybi, and use the Old Kingdom as a tax haven in order to save operating costs. After the
elaborate dinner, \Gls{Teppic} steps out onto the main deck, and falls asleep, overwhelmed by the
calm sea wind.

\subsubsection{\Gls{Teppic} learns about the founding of the Old Kingdom from \Gls{Khuft}}
In his dream, \Gls{Teppic} talks to his ancestor \Gls{Khuft}, who reveals the truth of how he
founded the kingdom, and tells him that the entire river valley just appeared all of a sudden. He
realizes that the entire lineage was based on a camel-trading conman running away from persecution
for his crimes, and wakes up to find \Gls{Chidder}'s boat leaving the harbour. He decides to take
\Gls{Khuft}'s words about his latent powers seriously, and dives off the deck, intending to return
to the Old Kingdom.

\subsubsection{\Gls{Teppic} plans to use \Gls{YouB} to get back to the Old Kingdom}
\Gls{Teppic} rescues \Gls{YouB} from an Ephebian stable, and tells him to head home. He thinks about
\Gls{Khuft}'s speculation that his large herd of camels was somehow responsible for the Old Kingdom
suddenly appearing in the desert. Meanwhile, \Gls{KingTep} rescues more of his ancestors from their
resting places, by having \Gls{Dil} and \Gls{Gern} break down the slabs sealing their pyramids.

\subsubsection{\Gls{Teppic} uses \Gls{YouB}'s computational power to open the portal to the
    Old Kingdom}
\Gls{Teppic} reaches the edge of the Old Kingdom, where some Ephebian soldiers have set up camp to
prepare for the war with Tsort. He tries to goad \Gls{YouB} into manifesting the river valley by
denying him water to drink, and asks him to focus on the river. \Gls{YouB} does the necessary
calculations in his mind, and manages to take \Gls{Teppic} through the dimensional portal in the
cliff to the Old Kingdom. The two armies gathered to watch the spectacle are left wondering if what
they witnessed was some kind of mirage.

\subsubsection{\Gls{Teppic} outmanoeuvres the \Gls{Sphinx} guarding the Old Kingdom}
\Gls{Teppic} realizes that \Gls{YouB} has brought him into a pocket dimension adjacent to that of
the Old Kingdom, and is stopped by the guardian \Gls{Sphinx}, armed with its riddle. \Gls{Teppic}
forces the \Gls{Sphinx} to reframe its riddle to make it mathematically consistent, and gives it the
same answer as before, while keeping it confused enough to not realize the question is essentially
the same.

\subsubsection{\Gls{Koomi} proceeds with his power-grab, as \Gls{Dios} fails to take charge of the
    crisis}
\Gls{YouB} correctly calculates the path to the Old Kingdom's pocket dimension, and takes
\Gls{Teppic} there, just as the \Gls{Sphinx} works out his deception. Meanwhile, the priests try
desperately to bend the Old Gods to their will using rituals, as \Gls{Dios} tries to think of a way
to placate the angry mob gathering outside the palace. \Gls{Koomi} reminds the priests of the old
practice that involved sacrificing a king to quell the anger of the Gods.

\subsubsection{\Gls{KingTep} inspects the pyramid of the first ancestor}
\Gls{Dios} resigns himself to death as per the rules he had authored many decades ago, as the other
priests spot the line of mummified kings walking along the opposite bank of the river. \Gls{KingTep}
asks \Gls{Dil} and \Gls{Gern} to accompany him with some torches, as they explore the smallest
pyramid in the necropolis, dedicated to the first ancestor. They find a small bed with a pillow
and blanket inside the main chamber, with no sign of a mummy.

\subsubsection{\Gls{Teppic} returns to the palace, and finds it empty}
\Gls{Gern} points out the notches scratched into the walls of \Gls{Khuft}'s chamber, as \Gls{Dil}
is horrified by the torch burning in reverse. Meanwhile, \Gls{Teppic} lets \Gls{YouB} rest at the
palace stables, and makes his way to the deserted throne room, finding the golden mask of the King
peeling away to reveal the lead underneath. He finds the citizens clustered around a jetty at the
Djel river, and proceeds to investigate.

\subsubsection{\Gls{Teppic} crosses the Djel river to get to the necropolis}
\Gls{Teppic} feels the powers of divinity flowing in his veins, strengthened by the citizens'
belief in the royal lineage. He realizes that the priests have taken every single boat in their
attempt to negotiate with the reanimated mummies, and parts the river using his newfound power
before walking across the bed.

\subsubsection{\Gls{KingTep} learns the identity of the first ancestor}
\Gls{Dil} and \Gls{Gern} set the line of ancestors in decreasing order of age, in order to decipher
the writings in the oldest pyramid. \Gls{KingTep} loses his cool when he finally hears the name of
the first ancestor, as it passes along the line of mummies, with many curses thrown in for good
measure. Meanwhile, \Gls{Teppic} notices the commoners buried in the necropolis slowly being freed
by groups of royalty, to then converge on the Great Pyramid

\subsubsection{Tsort and Ephebe prepare for the formal start of the war}
The armies of Tsort and Ephebe gather in force at the borders of Djelibeybi, and prepare their
wooden horses under cover of night. The generals remind themselves of the legends of the last war,
fought many centuries ago under the supervision of \Gls{Laveo}. Meanwhile, \Gls{Ptacl} watches the
priests walking in a line to the Great Pyramid, just behind the undead royal lineage of the Old
Kingdom.

\subsubsection{\Gls{KingTep} faces the priests, who appoint \Gls{Koomi} as their leader}
The Ephebian soldiers hide inside their wooden horse, fully convinced that the enemy will make the
same mistake it did many thousands of years ago. \Gls{Teppic} asks \Gls{Ptacl} if all of the excess
time stored in the Great Pyramid can be vented using a steel capstone, as a one-time process.
\Gls{Dios} tries to call the gathered ancestors foul demons, and is stopped by \Gls{KingTep}, who
lifts him off the ground with one hand.

\subsubsection{\Gls{Dios} confronts the dead ancestors of the royal family}
\Gls{Dios} stops the mummies from destroying the Great Pyramid, using the hidden power in his
staff. He confesses to being the same high priest that advised the first King seven thousand years
ago, keeping himself alive using the negative time flow in the small pyramid. \Gls{Dios} spots
\Gls{Teppic} scaling the Great Pyramid, and beckons \Gls{IIb} over to ask what the effects would be
if the time stored inside it is vented.

\subsubsection{\Gls{Dios} faces the Old Gods as they try to approach the Great Pyramid}
\Gls{Teppic} continues to climb, with \Gls{Koomi} trying to dissuade \Gls{Dios} from striking him
down using the magic staff. As the Old Gods start to sense something wrong at the Great Pyramid,
\Gls{Dios} is forced to turn his attention to them. He finds himself unable to control them, even
though he created them many centuries ago as a means of sociopolitical control over the citizens.

\subsubsection{\Gls{Teppic} reverses the dimensional anomaly caused by the Great Pyramid}
\Gls{Teppic} is carried to the top of the Great Pyramid as his ancestors form a human pyramid along
one of its faces to help him climb. Just as a God is about to squash him, \Gls{Teppic} places the
capstone on the Great Pyramid, and it explodes after releasing its pent up time, reversing the
banishment of the Old Kingdom. \Gls{Ptacl} desperately searches for his sons in the mist and rubble
caused by the explosion, until \Gls{IIb} calls out to him.

\subsubsection{The war abruptly ends when Djelibeybi comes back to the Disc}
\Gls{IIb} finds \Gls{Dil} in the rubble, and sets off to the crater where \Gls{Teppic} is lying
unconscious, with corn sprouting all around him. Meanwhile, an Ephebian soldier witnesses the
Djelibeybi empire pop back into the Disc, as he is getting ready to board the wooden horse prepared
as bait for the Tsorteans. \Gls{Chidder} drops in via ship to the palace, after the mummies have
vanished, and the workmen have started to clean up the ruined necropolis.

\subsubsection{\Gls{Teppic} manages to pass the burden of ruling the Old Kingdom to \Gls{Ptraci}}
\Gls{Teppic} welcomes \Gls{Chidder} and \Gls{Ptraci} to the court, and talks about abdicating his
throne in favour of any relatives that might still be alive. The handmaidens look into the records
and confirm that \Gls{Ptraci} is \Gls{Teppic}'s half-sister, when she mentions having the same
dream as he did regularly. He uses the opportunity to appoint her Queen, with \Gls{Koomi} herding
the priests into approving the declaration.

\subsubsection{\Gls{Ptraci} gets to work modernizing the Old Kingdom}
\Gls{Ptraci} immediately shrugs off the weight of tradition, leaving \Gls{Koomi} flummoxed as to his
role in the palace. He eventually comes to terms with his role as chief minister, and prepares to
deal with the two armies massed at the borders, waiting for permission to cross. Meanwhile,
\Gls{Ptacl} prepares to make some serious money building bridges across the Djel river, as \Gls{IIa}
thinks about the revenue to come from the newly commissioned port.

\subsubsection{\Gls{Ptraci} watches \Gls{Teppic} leave the kingdom, and becomes the uncontested
    ruler}
\Gls{Dil} and \Gls{Gern} think about new jobs, after \Gls{Ptraci} outlaws pyramids. Meanwhile,
\Gls{Death} oversees the processing of the thousands of royalty who are suddenly free from the
mortal realm. \Gls{Ptraci} comes down in person to try and stop \Gls{Teppic} from leaving, and is
forced to resign herself to being the sole ruler of the Kingdom. \Gls{Dios} is transported to the
start of the Old Kingdom, and sees \Gls{Khuft} reach the river with his herd of camels.